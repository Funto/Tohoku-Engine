% opengl_spaces.tex
% To be compiled with latex, not with pdflatex.

\documentclass{article}
\usepackage{pstricks}
\usepackage{pstricks-add}

\begin{document}

\title{OpenGL spaces and transformations}
\author{Lionel Fuentes}
\date{May 2010}
\maketitle

\section{Introduction}
This document is aimed at being a reminder for dealing with the different
3D spaces and transformations one has to work with when using the OpenGL
graphics API. I primarily made this document for myself, but feel free
to use it and distribute it as you want.

\section{OpenGL spaces}

%\begin{center}
\begin{tabular}{p{2cm} c}
	\begin{psmatrix}[rowsep=0.8,colsep=0.4]
		%~ [name=local] & \dianode[fillstyle=solid,fillcolor=red!30]{local}{Local coordinates}\\
					 %~ & \rnode{local_world}{\psframebox[fillstyle=solid,fillcolor=blue!30]{$model\times vertex$}}\\
		%~ [name=world] & \dianode[fillstyle=solid,fillcolor=red!30]{world}{World coordinates}\\
					 %~ & \rnode{world_eye}{\psframebox[fillstyle=solid,fillcolor=blue!30]{$view\times model\times vertex$}}\\
		%~ [name=eye]   & \dianode[fillstyle=solid,fillcolor=red!30]{eye}{Eye coordinates} & (or Camera space, or Eye space)\\

		\rnode{local}{\psframebox[fillstyle=solid,fillcolor=blue!30]{Local coordinates}}\\
		\rnode{world}{\psframebox[fillstyle=solid,fillcolor=blue!30]{World coordinates}}\\
		\rnode{eye}{\psframebox[fillstyle=solid,fillcolor=blue!30]{Eye coordinates (=camera space = eye space)}} & Light calculations\\
	\end{psmatrix}

	\psset{arrows=->,nodesep=0pt}
	%~ \ncline{local}{local_world}
	%~ \ncline{local_world}{world}
	%~ \ncline{world}{world_eye}
	%~ \ncline{world_eye}{eye}

	\ncline{local}{world}\nbput{$model\times vertex$}
	\ncline{world}{eye}\nbput{$view\times model\times vertex$}

	%\psbrace[ref=r,rot=180,nodesepA=0pt](world)(eye){Test}

	%\ncline{local}{local_world}\nbput{Yes}
	%\psbrace[ref=r,rot=180,nodesepA=-8pt](X1)(X2){Stage 1}
	%\psbrace[ref=r,rot=180,nodesepA=-8pt](X2)(Y2){\parbox{2cm}{Stage 2 very looooooong text}}
\end{tabular}
%\end{center}

\end{document}

%~ \documentclass{article}
%~ \usepackage{pstricks}
%~ \usepackage{pstricks-add}
%~
%~ \begin{document}
%~
%~ \begin{center}
%~ \begin{tabular}{p{2cm} c}
   %~ \begin{psmatrix}[rowsep=0.8,colsep=0.4]
     %~ [name=X1] & \\[-25pt]
               %~ & \dianode[fillstyle=solid,fillcolor=red!30]{S}{Condition1?}\\[0pt]
               %~ & \rnode{SA}{\psframebox[fillstyle=solid,fillcolor=blue!30]{do something}}\\[-25pt]
     %~ [name=X2] & \\[0pt]
               %~ & \dianode[fillstyle=solid,fillcolor=red!30]{R}{Condition2?}\\[-10pt]
     %~ [name=Y2] &
   %~ \end{psmatrix}%
       %~ \psset{arrows=->,nodesep=0pt}
       %~ \ncline{S}{SA}\nbput{Yes}
       %~ \ncline{SA}{R}
     %~ \psbrace[ref=r,rot=180,nodesepA=-8pt](X1)(X2){Stage 1}
     %~ \psbrace[ref=r,rot=180,nodesepA=-8pt](X2)(Y2){\parbox{2cm}{Stage 2 very
       %~ looooooong text}}
%~ \end{tabular}
%~ \end{center}
%~
%~ \end{document}
